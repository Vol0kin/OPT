\documentclass[11pt,a4paper]{article}
\usepackage[english]{babel}					% Use english
\usepackage[utf8]{inputenc}					% Caracteres UTF-8
\usepackage{graphicx}						% Imagenes
\usepackage[hidelinks]{hyperref}			% Poner enlaces sin marcarlos en rojo
\usepackage{fancyhdr}						% Modificar encabezados y pies de pagina
\usepackage{float}							% Insertar figuras
\usepackage[textwidth=390pt]{geometry}		% Anchura de la pagina
\usepackage[nottoc]{tocbibind}				% Referencias (no incluir num pagina indice en Indice)
\usepackage{enumitem}						% Permitir enumerate con distintos simbolos
\usepackage[T1]{fontenc}					% Usar textsc en sections
\usepackage{amsmath}						% Símbolos matemáticos
\usepackage{amssymb}

% Comando para poner el nombre de la asignatura
\newcommand{\subject}{Optimization}
\newcommand{\autor}{Vladislav Nikolov Vasilev}
\newcommand{\titulo}{Optimization Problem 10}
\newcommand{\subtitulo}{Simplex problem}
\newcommand{\masters}{Master in Fundamental Principles of Data Science}

% Configuracion de encabezados y pies de pagina
\pagestyle{fancy}
\lhead{\autor{}}
\rhead{\subject{}}
\lfoot{\masters}
\cfoot{}
\rfoot{\thepage}
\renewcommand{\headrulewidth}{0.4pt}		% Linea cabeza de pagina
\renewcommand{\footrulewidth}{0.4pt}		% Linea pie de pagina

\begin{document}
\pagenumbering{gobble}

% Title page
\begin{titlepage}
  \begin{minipage}{\textwidth}
    \centering
    \includegraphics[scale=0.25]{img/ub-logo}\\[2cm]
    
    \textsc{\Large \subject\\[0.5cm]}
    \textsc{\uppercase\expandafter{\masters}}\\[1.5cm]
    
    \noindent\rule[-1ex]{\textwidth}{1pt}\\[1.5ex]
    \textsc{{\Huge \titulo\\[0.5ex]}}
    \textsc{{\Large \subtitulo\\}}
    \noindent\rule[-1ex]{\textwidth}{2pt}\\[3.5ex]
  \end{minipage}
  
  \vspace{2cm}
  
  \begin{minipage}{\textwidth}
    \centering
    
    \includegraphics[scale=0.4]{img/ub-ds-logo}
    \vspace{2cm}
    
    \textbf{Author}\\ {\autor{}}\\[2.5ex]
    \textsc{Faculty of Mathematics and Computer Science}\\
    \vspace{1em}
    \textsc{Academic year 2021-2022}
  \end{minipage}
\end{titlepage}

\pagenumbering{arabic}
\setlength{\parskip}{1em}


\section{Problem description}

Prove that the number of faces of dimension $p$ of a $n$-dimensional simplex
is equal to

\[
\begin{pmatrix}
  n + 1 \\
  p + 1
\end{pmatrix}
=
\frac{(n+1)!}{(p+1)!(n-p)!}
\]

\section{Solution}

By definition, a $n$-dimensional simplex $S$ is a convex polyhedron with $n + 1$ vertices.
Moreover, we have that the faces of a simplex are a lower dimensional simplex and that $0 \leq p \leq n$.
Therefore, a face of dimension $p$ of a $n$-dimensional is a simplex made up by $p + 1$ vertices
of the $n + 1$ vertices from the original simplex.

The number of faces of dimension $p$ is given by the amount of \textbf{combinations} of $p + 1$ vertices
that we can make by selecting them among the $n + 1$ available vertices. We can express the
previous statement using combinatorics, which results in the following expression:

\[
  C^{n+1}_{p+1} =
 \begin{pmatrix}
    n + 1 \\
    p + 1
  \end{pmatrix}
  =
  \frac{(n+1)!}{(p+1)!\big((n+1) - (p+1)\big)!}
  =
  \frac{(n+1)!}{(p+1)!(n-p)!}
\]

\end{document}
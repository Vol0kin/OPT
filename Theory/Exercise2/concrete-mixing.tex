\documentclass[11pt,a4paper]{article}
\usepackage[english]{babel}					% Use english
\usepackage[utf8]{inputenc}					% Caracteres UTF-8
\usepackage{graphicx}						% Imagenes
\usepackage[hidelinks]{hyperref}			% Poner enlaces sin marcarlos en rojo
\usepackage{fancyhdr}						% Modificar encabezados y pies de pagina
\usepackage{float}							% Insertar figuras
\usepackage[textwidth=390pt]{geometry}		% Anchura de la pagina
\usepackage[nottoc]{tocbibind}				% Referencias (no incluir num pagina indice en Indice)
\usepackage{enumitem}						% Permitir enumerate con distintos simbolos
\usepackage[T1]{fontenc}					% Usar textsc en sections
\usepackage{amsmath}						% Símbolos matemáticos
\usepackage{amssymb}

% Comando para poner el nombre de la asignatura
\newcommand{\subject}{Optimization}
\newcommand{\autor}{Vladislav Nikolov Vasilev}
\newcommand{\titulo}{Optimization problem 2}
\newcommand{\subtitulo}{Concrete mixing problem}
\newcommand{\masters}{Master in Fundamental Principles of Data Science}

\newcommand{\norm}[1]{\left\lVert#1\right\rVert}
\newcommand{\algebraVector}[1]{\boldsymbol{#1}}


% Configuracion de encabezados y pies de pagina
\pagestyle{fancy}
\lhead{\autor{}}
\rhead{\subject{}}
\lfoot{\masters}
\cfoot{}
\rfoot{\thepage}
\renewcommand{\headrulewidth}{0.4pt}		% Linea cabeza de pagina
\renewcommand{\footrulewidth}{0.4pt}		% Linea pie de pagina

\begin{document}
\pagenumbering{gobble}

% Title page
\begin{titlepage}
  \begin{minipage}{\textwidth}
    \centering
    \includegraphics[scale=0.25]{img/ub-logo}\\[2cm]
    
    \textsc{\Large \subject\\[0.5cm]}
    \textsc{\uppercase\expandafter{\masters}}\\[1.5cm]
    
    \noindent\rule[-1ex]{\textwidth}{1pt}\\[1.5ex]
    \textsc{{\Huge \titulo\\[0.5ex]}}
    \textsc{{\Large \subtitulo\\}}
    \noindent\rule[-1ex]{\textwidth}{2pt}\\[3.5ex]
  \end{minipage}
  
  \vspace{2cm}
  
  \begin{minipage}{\textwidth}
    \centering
    
    \includegraphics[scale=0.4]{img/ub-ds-logo}
    \vspace{2cm}
    
    \textbf{Author}\\ {\autor{}}\\[2.5ex]
    \textsc{Faculty of Mathematics and Computer Science}\\
    \vspace{1em}
    \textsc{Academic year 2021-2022}
  \end{minipage}
\end{titlepage}

\pagenumbering{arabic}
\setlength{\parskip}{1em}

\section{Problem description}

\emph{\textbf{Suppose that we are mixing concrete and are using $n$ different gravel sizes
$s_1, \dots, s_n$.}}

\emph{\textbf{The ideal mixture is given by $\algebraVector{c} = (c_1, \dots, c_n)$, where
$c_i$ $(0 \leq c_i \leq 1)$ is the fraction of size $s_i$ in the mix, and $\sum_{i=1}^n c_i = 1$.}}

\emph{\textbf{Gravel mixtures come from $m$ different mines. The gravel
composition at each mine $j = 1, \dots, m$ is given by $C_j = (c_{1j}, \dots, c_{nj})$, where
$0 \leq c_{ij} \leq 1$ for all $i = 1, \dots, n$ and $\sum_{i=1}^n c_{ij} = 1$.}}

\emph{\textbf{Let $\algebraVector{x} = (x_1, \dots, x_m)$ be the vector that represents the
fraction of gravel of the mines in the mixture, where $0 \leq x_j \leq 1$ for all
$j = 1, \dots, m$ and $\sum_{j=1}^m x_j = 1$.}}

\emph{\textbf{Find the best possible approximation $\algebraVector{x} = (x_1, \dots, x_m)$
of the ideal mixture, $\algebraVector{c} = (c_1, \dots, c_n)$, by using the material from
the $m$ mines.}}

\section{Solution}

We have that $\algebraVector{x} \in \mathbb{R}^m$, $\algebraVector{c} \in \mathbb{R}^n$ and
$C \in \mathbb{R}^{n \times m}$, where the matrix $C = (C_1, \dots, C_m)$ has $C_j$ as columns.

Finding the best possible vector $\algebraVector{x}$ means finding a vector such that

\begin{equation}
  \label{eq:original-problem}
  C\algebraVector{x} = \algebraVector{c}
\end{equation}

This means that the vector result of the matrix-vector product $C\algebraVector{x}$ has to be
as similar as possible to $\algebraVector{c}$. With this in mind, we can rewrite expression
\eqref{eq:original-problem} as the following optimization problem:

\begin{subequations}
\begin{alignat}{2}
  \algebraVector{x}^\star &= \min_{x \in \mathbb{R}^m} \norm{C\algebraVector{x} - \algebraVector{c}}^2 \label{eq:minimization-problem} \\
  &\text{subjet to } \sum_{j=1}^m x_j = 1 \text{, and } x_j \geq 0 \label{eq:restrictions}
\end{alignat}
\end{subequations}

\noindent where $\algebraVector{x}^\star$ is the best possible solution out of all the
feasible ones. Note that minimizing $\norm{C\algebraVector{x} - \algebraVector{c}}^2$
is the same as minimizing $\norm{C\algebraVector{x} - \algebraVector{c}}$.

Now, knowing that $\norm{\algebraVector{x}} = \sqrt{\algebraVector{x}^T\algebraVector{x}}$, the
distance found in expression \eqref{eq:minimization-problem} can be rewritten as follows:

\begin{equation}
  \label{eq:quadratic-opt-problem}
\begin{aligned}
  \norm{C\algebraVector{x} - \algebraVector{c}}^2 &=
  (C\algebraVector{x} - \algebraVector{c})^T (C\algebraVector{x} - \algebraVector{c})\\
  &= \algebraVector{x}^T C^T C \algebraVector{x} - \algebraVector{x}^T C^T \algebraVector{c} - \algebraVector{c}^T C \algebraVector{x} + \algebraVector{c}^T \algebraVector{c} \\
  & = \algebraVector{x}^T C^T C \algebraVector{x} - (\algebraVector{x}^T C^T \algebraVector{c})^T - \algebraVector{c}^T C \algebraVector{x} + \algebraVector{c}^T \algebraVector{c} \\
  & = \algebraVector{x}^T C^T C \algebraVector{x} - \algebraVector{c}^T C \algebraVector{x} - \algebraVector{c}^T C \algebraVector{x} + \algebraVector{c}^T \algebraVector{c} \\
  & = \algebraVector{x}^T C^T C \algebraVector{x} - 2\algebraVector{c}^T C \algebraVector{x} + \norm{\algebraVector{c}}^2
\end{aligned}
\end{equation}

Thanks to the last expression in \eqref{eq:quadratic-opt-problem}, we can clearly see that
this is a \textbf{Quadratic Optimization} problem because the objective function is quadratic and
all the restrictions are linear.

Therefore, finding a value of $\algebraVector{x}$ which minimizes the expression
\eqref{eq:minimization-problem} and satisfies the restrictions seen in expression
\eqref{eq:restrictions} is guaranteed to give us a good approximation of the ideal mixture.
\end{document}


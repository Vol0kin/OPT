\documentclass[11pt,a4paper]{article}
\usepackage[english]{babel}					% Use english
\usepackage[utf8]{inputenc}					% Caracteres UTF-8
\usepackage{graphicx}						% Imagenes
\usepackage[hidelinks]{hyperref}			% Poner enlaces sin marcarlos en rojo
\usepackage{fancyhdr}						% Modificar encabezados y pies de pagina
\usepackage{float}							% Insertar figuras
\usepackage[textwidth=390pt]{geometry}		% Anchura de la pagina
\usepackage[nottoc]{tocbibind}				% Referencias (no incluir num pagina indice en Indice)
\usepackage{enumitem}						% Permitir enumerate con distintos simbolos
\usepackage[T1]{fontenc}					% Usar textsc en sections
\usepackage{amsmath}						% Símbolos matemáticos
\usepackage{amssymb}

% Comando para poner el nombre de la asignatura
\newcommand{\subject}{Optimization}
\newcommand{\autor}{Vladislav Nikolov Vasilev}
\newcommand{\titulo}{Optimization Problem 3}
\newcommand{\subtitulo}{Convex Function problem}
\newcommand{\masters}{Master in Fundamental Principles of Data Science}

% Configuracion de encabezados y pies de pagina
\pagestyle{fancy}
\lhead{\autor{}}
\rhead{\subject{}}
\lfoot{\masters}
\cfoot{}
\rfoot{\thepage}
\renewcommand{\headrulewidth}{0.4pt}		% Linea cabeza de pagina
\renewcommand{\footrulewidth}{0.4pt}		% Linea pie de pagina

\begin{document}
\pagenumbering{gobble}

% Title page
\begin{titlepage}
  \begin{minipage}{\textwidth}
    \centering
    \includegraphics[scale=0.25]{img/ub-logo}\\[2cm]
    
    \textsc{\Large \subject\\[0.5cm]}
    \textsc{\uppercase\expandafter{\masters}}\\[1.5cm]
    
    \noindent\rule[-1ex]{\textwidth}{1pt}\\[1.5ex]
    \textsc{{\Huge \titulo\\[0.5ex]}}
    \textsc{{\Large \subtitulo\\}}
    \noindent\rule[-1ex]{\textwidth}{2pt}\\[3.5ex]
  \end{minipage}
  
  \vspace{2cm}
  
  \begin{minipage}{\textwidth}
    \centering
    
    \includegraphics[scale=0.4]{img/ub-ds-logo}
    \vspace{2cm}
    
    \textbf{Author}\\ {\autor{}}\\[2.5ex]
    \textsc{Faculty of Mathematics and Computer Science}\\
    \vspace{1em}
    \textsc{Academic year 2021-2022}
  \end{minipage}
\end{titlepage}

\pagenumbering{arabic}
\setlength{\parskip}{1em}


\section{Problem description}

\emph{\textbf{Prove, without using the above theorem, that for any
$a \in \mathbb{R}$, $f(x) = e^{ax}$ is a convex function.}}

\section{Solution}

To prove that the function $f(x) = e^{ax}$ is a convex function for any $a \in \mathbb{R}$,
let's first suppose that we have two points, $x$ and $y$, such that $x, y \in \mathbb{R}$ and $x \neq y$.
For the function $f(x)$ to be convex, the following inequality has to be satisfied:

\begin{equation}
  \label{eq:convex-function-inequality}
  f(\lambda x + (1 - \lambda)y) \leq \lambda f(x) + (1 - \lambda)f(y), \quad \forall \lambda \in [0, 1]
\end{equation}

Now, let us substitute $f$ in expression \eqref{eq:convex-function-inequality} with our function:

\begin{equation}
  \label{eq:exponential-inequality}
  e^{a(\lambda x + (1 - \lambda)y)} \leq \lambda e^{ax} + (1 - \lambda)e^{ay}
\end{equation}

If we further operate, we obtain the following:

\begin{equation}
  \label{eq:exponential-operations}
  \begin{aligned}[b]
    e^{\lambda ax + ay - \lambda ay} &\leq \lambda e^{ax} + e^{ay} - \lambda e^{ay} \\
    e^{\lambda ax} e^{ay} e^{-\lambda ay} &\leq \lambda e^{ax} + e^{ay} - \lambda e^{ay} \\
    e^{\lambda ax} e^{-\lambda ay} &\leq \lambda \frac{e^{ax}}{e^{ay}} + 1 - \lambda \\
    e^{\lambda ax - \lambda ay} &\leq \lambda e^{ax - ay} + 1 - \lambda \\
    e^{\lambda a(x - y)} &\leq \lambda e^{a(x - y)} + 1 - \lambda
  \end{aligned}
\end{equation}

To simplify a little bit the previous expression, let us introduce a new variable, $z$,
such that $z = a(x - y)$. If we substitute accordingly in the above result, we obtain the
following:

\begin{equation}
  \label{eq:exponential-substitution}
  e^{\lambda z} \leq \lambda e^{z} + 1 - \lambda
\end{equation}

Knowing that $e^x$ can be defined as a power series as:

$$e^x = \sum_{n=0}^\infty \frac{x^n}{n!} = 1 + x + \frac{x^2}{2!} + \frac{x^3}{3!} + \dots$$

\noindent we can substitute $e^x$ in the expression \eqref{eq:exponential-substitution}. After
doing that, we get:

\begin{equation}
  \label{eq:power-series-inequality}
  \begin{aligned}[b]
    1 + \lambda z + \frac{(\lambda z)^2}{2!} + \frac{(\lambda z)^3}{3!} + \dots &\leq
    \lambda \bigg( 1 + z + \frac{(z)^2}{2!} + \frac{(z)^3}{3!} + \dots \bigg) + 1 - \lambda \\
    \lambda + \lambda z + \frac{(\lambda z)^2}{2!} + \frac{(\lambda z)^3}{3!} + \dots &\leq
    \lambda \bigg( 1 + z + \frac{z^2}{2!} + \frac{z^3}{3!} + \dots \bigg) \\
    \lambda \bigg( 1 + z + \frac{\lambda z^2}{2!} + \frac{\lambda^2 z^3}{3!} + \dots \bigg) &\leq
    \lambda \bigg( 1 + z + \frac{z^2}{2!} + \frac{z^3}{3!} + \dots \bigg) \\
    1 + z + \frac{\lambda z^2}{2!} + \frac{\lambda^2 z^3}{3!} + \dots &\leq
    1 + z + \frac{z^2}{2!} + \frac{z^3}{3!} + \dots
  \end{aligned}
\end{equation}

With the result seen in expression \eqref{eq:power-series-inequality}, we can distinguish
3 different cases:

\begin{itemize}
  \item \textbf{Case $a = 0$}: It is easy to see that by substituting $a = 0$ in the above
  result, we get that $1 \leq 1$, which satisfies the inequality.
  \item \textbf{Case $a > 0$}: In this case, we can select $x$ and $y$ accordingly so that
  $z$ is positive, which will be true when $x > y$. If we then select any value of $\lambda$ such
  that $0 \leq \lambda \leq 1$, we can clearly see that the right expression will always be
  smaller than the left one because the addends on the left get scaled down by $\lambda$.
  Therefore, the inequality will be true.
  \item \textbf{Case $a < 0$}: Here we can also select the points $x$ and $y$ accordingly
  so that the value of $z$ is positive. Since $a$ is negative, the subtraction has to be negative,
  which will be true when $x < y$\footnote{Note that if we are using the same points as in the previous
  case, we are just swapping their values. Because of this, we cannot use the same value of
  $\lambda$ as before. Therefore, we have to use $\lambda'$, which is given by the expression
  $\lambda' = 1 - \lambda$.}. 
  Since $z$ is positive again, we get the same result as in the previous case. Thus, the inequality
  is satisfied again.
\end{itemize}

As we can see, the inequality defined in expression \eqref{eq:power-series-inequality} is satisfied
for any $a \in \mathbb{R}$. Thus, we can conclude that the function $f(x) = e^{ax}$ is convex for
any value of $a$.

\end{document}
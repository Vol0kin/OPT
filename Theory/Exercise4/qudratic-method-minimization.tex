\documentclass[11pt,a4paper]{article}
\usepackage[english]{babel}					% Use english
\usepackage[utf8]{inputenc}					% Caracteres UTF-8
\usepackage{graphicx}						% Imagenes
\usepackage[hidelinks]{hyperref}			% Poner enlaces sin marcarlos en rojo
\usepackage{fancyhdr}						% Modificar encabezados y pies de pagina
\usepackage{float}							% Insertar figuras
\usepackage[textwidth=390pt]{geometry}		% Anchura de la pagina
\usepackage[nottoc]{tocbibind}				% Referencias (no incluir num pagina indice en Indice)
\usepackage{enumitem}						% Permitir enumerate con distintos simbolos
\usepackage[T1]{fontenc}					% Usar textsc en sections
\usepackage{amsmath}						% Símbolos matemáticos
\usepackage{amssymb}

% Comando para poner el nombre de la asignatura
\newcommand{\subject}{Optimization}
\newcommand{\autor}{Vladislav Nikolov Vasilev}
\newcommand{\titulo}{Optimization Problem 4}
\newcommand{\subtitulo}{Quadratic method minimization problem}
\newcommand{\masters}{Master in Fundamental Principles of Data Science}

% Configuracion de encabezados y pies de pagina
\pagestyle{fancy}
\lhead{\autor{}}
\rhead{\subject{}}
\lfoot{\masters}
\cfoot{}
\rfoot{\thepage}
\renewcommand{\headrulewidth}{0.4pt}		% Linea cabeza de pagina
\renewcommand{\footrulewidth}{0.4pt}		% Linea pie de pagina

\begin{document}
\pagenumbering{gobble}

% Title page
\begin{titlepage}
  \begin{minipage}{\textwidth}
    \centering
    \includegraphics[scale=0.25]{img/ub-logo}\\[2cm]
    
    \textsc{\Large \subject\\[0.5cm]}
    \textsc{\uppercase\expandafter{\masters}}\\[1.5cm]
    
    \noindent\rule[-1ex]{\textwidth}{1pt}\\[1.5ex]
    \textsc{{\Huge \titulo\\[0.5ex]}}
    \textsc{{\Large \subtitulo\\}}
    \noindent\rule[-1ex]{\textwidth}{2pt}\\[3.5ex]
  \end{minipage}
  
  \vspace{2cm}
  
  \begin{minipage}{\textwidth}
    \centering
    
    \includegraphics[scale=0.4]{img/ub-ds-logo}
    \vspace{2cm}
    
    \textbf{Author}\\ {\autor{}}\\[2.5ex]
    \textsc{Faculty of Mathematics and Computer Science}\\
    \vspace{1em}
    \textsc{Academic year 2021-2022}
  \end{minipage}
\end{titlepage}

\pagenumbering{arabic}
\setlength{\parskip}{1em}


\section{Problem description}

Let $f$ be a real function on $\mathbb{R}^n$. Also let $x_0 \in \mathbb{R}^n$,
$z \in \mathbb{R}^n$, and $\theta \in \mathbb{R}$. Define

$$F(\theta) = f(x_0 + \theta z)$$

\noindent and suppose that we are looking for the minimum of $F$ (that is, for the
minimum of $f$ in the direction $z$ through the point $x_0$). Let $x_0 + \theta_1z$, 
$x_0 + \theta_2z$ and $x_0 + \theta_3z$ be three points where $f$ is evaluated.
Show that the minimum predicted by applying the quadratic approximation method is
$x_0 + \theta^* z$, where

$$ \theta^* = \frac{[\theta_2^2 - \theta_3^2]F(\theta_1) + [\theta_3^2 - \theta_1^2]F(\theta_2) + [\theta_1^2 - \theta_2^2]F(\theta_3)}{2[(\theta_2 - \theta_3)F(\theta_1) + (\theta_3 - \theta_1)F(\theta_2) + (\theta_1 - \theta_2)F(\theta_3)]}$$

\noindent and it is indeed the minimum of the parabola passing through the above three
points if

$$\frac{(\theta_2 - \theta_3)F(\theta_1) + (\theta_3 - \theta_1)F(\theta_2) + (\theta_1 - \theta_2)F(\theta_3)}{(\theta_2 - \theta_3)(\theta_3 - \theta_1)(\theta_1 - \theta_2)} < 0$$

\section{Solution}


\end{document}
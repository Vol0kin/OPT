\documentclass[11pt]{article}

    \usepackage[breakable]{tcolorbox}
    \usepackage{parskip} % Stop auto-indenting (to mimic markdown behaviour)
    \usepackage{fancyhdr}						% Modificar encabezados y pies de pagina

    \usepackage{iftex}
    \ifPDFTeX
    	\usepackage[T1]{fontenc}
    	\usepackage{mathpazo}
    \else
    	\usepackage{fontspec}
    \fi

    % Basic figure setup, for now with no caption control since it's done
    % automatically by Pandoc (which extracts ![](path) syntax from Markdown).
    \usepackage{graphicx}
    % Maintain compatibility with old templates. Remove in nbconvert 6.0
    \let\Oldincludegraphics\includegraphics
    % Ensure that by default, figures have no caption (until we provide a
    % proper Figure object with a Caption API and a way to capture that
    % in the conversion process - todo).
    \usepackage{caption}
    \DeclareCaptionFormat{nocaption}{}
    \captionsetup{format=nocaption,aboveskip=0pt,belowskip=0pt}

    \usepackage{float}
    \floatplacement{figure}{H} % forces figures to be placed at the correct location
    \usepackage{xcolor} % Allow colors to be defined
    \usepackage{enumerate} % Needed for markdown enumerations to work
    \usepackage[textwidth=390pt]{geometry} % Used to adjust the document margins
    \usepackage{amsmath} % Equations
    \usepackage{amssymb} % Equations
    \usepackage{textcomp} % defines textquotesingle
    % Hack from http://tex.stackexchange.com/a/47451/13684:
    \AtBeginDocument{%
        \def\PYZsq{\textquotesingle}% Upright quotes in Pygmentized code
    }
    \usepackage{upquote} % Upright quotes for verbatim code
    \usepackage{eurosym} % defines \euro
    \usepackage[mathletters]{ucs} % Extended unicode (utf-8) support
    \usepackage{fancyvrb} % verbatim replacement that allows latex
    \usepackage{grffile} % extends the file name processing of package graphics 
                         % to support a larger range
    \makeatletter % fix for old versions of grffile with XeLaTeX
    \@ifpackagelater{grffile}{2019/11/01}
    {
      % Do nothing on new versions
    }
    {
      \def\Gread@@xetex#1{%
        \IfFileExists{"\Gin@base".bb}%
        {\Gread@eps{\Gin@base.bb}}%
        {\Gread@@xetex@aux#1}%
      }
    }
    \makeatother
    \usepackage[Export]{adjustbox} % Used to constrain images to a maximum size
    \adjustboxset{max size={0.9\linewidth}{0.9\paperheight}}

    % The hyperref package gives us a pdf with properly built
    % internal navigation ('pdf bookmarks' for the table of contents,
    % internal cross-reference links, web links for URLs, etc.)
    \usepackage{hyperref}
    % The default LaTeX title has an obnoxious amount of whitespace. By default,
    % titling removes some of it. It also provides customization options.
    \usepackage{titling}
    \usepackage{longtable} % longtable support required by pandoc >1.10
    \usepackage{booktabs}  % table support for pandoc > 1.12.2
    \usepackage[inline]{enumitem} % IRkernel/repr support (it uses the enumerate* environment)
    \usepackage[normalem]{ulem} % ulem is needed to support strikethroughs (\sout)
                                % normalem makes italics be italics, not underlines
    \usepackage{mathrsfs}
    

    
    % Colors for the hyperref package
    \definecolor{urlcolor}{rgb}{0,.145,.698}
    \definecolor{linkcolor}{rgb}{.71,0.21,0.01}
    \definecolor{citecolor}{rgb}{.12,.54,.11}

    % ANSI colors
    \definecolor{ansi-black}{HTML}{3E424D}
    \definecolor{ansi-black-intense}{HTML}{282C36}
    \definecolor{ansi-red}{HTML}{E75C58}
    \definecolor{ansi-red-intense}{HTML}{B22B31}
    \definecolor{ansi-green}{HTML}{00A250}
    \definecolor{ansi-green-intense}{HTML}{007427}
    \definecolor{ansi-yellow}{HTML}{DDB62B}
    \definecolor{ansi-yellow-intense}{HTML}{B27D12}
    \definecolor{ansi-blue}{HTML}{208FFB}
    \definecolor{ansi-blue-intense}{HTML}{0065CA}
    \definecolor{ansi-magenta}{HTML}{D160C4}
    \definecolor{ansi-magenta-intense}{HTML}{A03196}
    \definecolor{ansi-cyan}{HTML}{60C6C8}
    \definecolor{ansi-cyan-intense}{HTML}{258F8F}
    \definecolor{ansi-white}{HTML}{C5C1B4}
    \definecolor{ansi-white-intense}{HTML}{A1A6B2}
    \definecolor{ansi-default-inverse-fg}{HTML}{FFFFFF}
    \definecolor{ansi-default-inverse-bg}{HTML}{000000}

    % common color for the border for error outputs.
    \definecolor{outerrorbackground}{HTML}{FFDFDF}

    % commands and environments needed by pandoc snippets
    % extracted from the output of `pandoc -s`
    \providecommand{\tightlist}{%
      \setlength{\itemsep}{0pt}\setlength{\parskip}{0pt}}
    \DefineVerbatimEnvironment{Highlighting}{Verbatim}{commandchars=\\\{\}}
    % Add ',fontsize=\small' for more characters per line
    \newenvironment{Shaded}{}{}
    \newcommand{\KeywordTok}[1]{\textcolor[rgb]{0.00,0.44,0.13}{\textbf{{#1}}}}
    \newcommand{\DataTypeTok}[1]{\textcolor[rgb]{0.56,0.13,0.00}{{#1}}}
    \newcommand{\DecValTok}[1]{\textcolor[rgb]{0.25,0.63,0.44}{{#1}}}
    \newcommand{\BaseNTok}[1]{\textcolor[rgb]{0.25,0.63,0.44}{{#1}}}
    \newcommand{\FloatTok}[1]{\textcolor[rgb]{0.25,0.63,0.44}{{#1}}}
    \newcommand{\CharTok}[1]{\textcolor[rgb]{0.25,0.44,0.63}{{#1}}}
    \newcommand{\StringTok}[1]{\textcolor[rgb]{0.25,0.44,0.63}{{#1}}}
    \newcommand{\CommentTok}[1]{\textcolor[rgb]{0.38,0.63,0.69}{\textit{{#1}}}}
    \newcommand{\OtherTok}[1]{\textcolor[rgb]{0.00,0.44,0.13}{{#1}}}
    \newcommand{\AlertTok}[1]{\textcolor[rgb]{1.00,0.00,0.00}{\textbf{{#1}}}}
    \newcommand{\FunctionTok}[1]{\textcolor[rgb]{0.02,0.16,0.49}{{#1}}}
    \newcommand{\RegionMarkerTok}[1]{{#1}}
    \newcommand{\ErrorTok}[1]{\textcolor[rgb]{1.00,0.00,0.00}{\textbf{{#1}}}}
    \newcommand{\NormalTok}[1]{{#1}}
    
    % Additional commands for more recent versions of Pandoc
    \newcommand{\ConstantTok}[1]{\textcolor[rgb]{0.53,0.00,0.00}{{#1}}}
    \newcommand{\SpecialCharTok}[1]{\textcolor[rgb]{0.25,0.44,0.63}{{#1}}}
    \newcommand{\VerbatimStringTok}[1]{\textcolor[rgb]{0.25,0.44,0.63}{{#1}}}
    \newcommand{\SpecialStringTok}[1]{\textcolor[rgb]{0.73,0.40,0.53}{{#1}}}
    \newcommand{\ImportTok}[1]{{#1}}
    \newcommand{\DocumentationTok}[1]{\textcolor[rgb]{0.73,0.13,0.13}{\textit{{#1}}}}
    \newcommand{\AnnotationTok}[1]{\textcolor[rgb]{0.38,0.63,0.69}{\textbf{\textit{{#1}}}}}
    \newcommand{\CommentVarTok}[1]{\textcolor[rgb]{0.38,0.63,0.69}{\textbf{\textit{{#1}}}}}
    \newcommand{\VariableTok}[1]{\textcolor[rgb]{0.10,0.09,0.49}{{#1}}}
    \newcommand{\ControlFlowTok}[1]{\textcolor[rgb]{0.00,0.44,0.13}{\textbf{{#1}}}}
    \newcommand{\OperatorTok}[1]{\textcolor[rgb]{0.40,0.40,0.40}{{#1}}}
    \newcommand{\BuiltInTok}[1]{{#1}}
    \newcommand{\ExtensionTok}[1]{{#1}}
    \newcommand{\PreprocessorTok}[1]{\textcolor[rgb]{0.74,0.48,0.00}{{#1}}}
    \newcommand{\AttributeTok}[1]{\textcolor[rgb]{0.49,0.56,0.16}{{#1}}}
    \newcommand{\InformationTok}[1]{\textcolor[rgb]{0.38,0.63,0.69}{\textbf{\textit{{#1}}}}}
    \newcommand{\WarningTok}[1]{\textcolor[rgb]{0.38,0.63,0.69}{\textbf{\textit{{#1}}}}}
    
    
    % Define a nice break command that doesn't care if a line doesn't already
    % exist.
    \def\br{\hspace*{\fill} \\* }
    % Math Jax compatibility definitions
    \def\gt{>}
    \def\lt{<}
    \let\Oldtex\TeX
    \let\Oldlatex\LaTeX
    \renewcommand{\TeX}{\textrm{\Oldtex}}
    \renewcommand{\LaTeX}{\textrm{\Oldlatex}}
    % Document parameters
    % Document title
    % Comando para poner el nombre de la asignatura
\newcommand{\subject}{Optimization}
\newcommand{\autor}{Vladislav Nikolov Vasilev}
\newcommand{\titulo}{Optimization Problem 6}
\newcommand{\subtitulo}{Conjugate gradient method problem II}
\newcommand{\masters}{Master in Fundamental Principles of Data Science}

% Configuracion de encabezados y pies de pagina
\pagestyle{fancy}
\lhead{\autor{}}
\rhead{\subject{}}
\lfoot{\masters}
\cfoot{}
\rfoot{\thepage}
\renewcommand{\headrulewidth}{0.4pt}		% Linea cabeza de pagina
\renewcommand{\footrulewidth}{0.4pt}		% Linea pie de pagina
    
    
    
    
    
% Pygments definitions
\makeatletter
\def\PY@reset{\let\PY@it=\relax \let\PY@bf=\relax%
    \let\PY@ul=\relax \let\PY@tc=\relax%
    \let\PY@bc=\relax \let\PY@ff=\relax}
\def\PY@tok#1{\csname PY@tok@#1\endcsname}
\def\PY@toks#1+{\ifx\relax#1\empty\else%
    \PY@tok{#1}\expandafter\PY@toks\fi}
\def\PY@do#1{\PY@bc{\PY@tc{\PY@ul{%
    \PY@it{\PY@bf{\PY@ff{#1}}}}}}}
\def\PY#1#2{\PY@reset\PY@toks#1+\relax+\PY@do{#2}}

\@namedef{PY@tok@w}{\def\PY@tc##1{\textcolor[rgb]{0.73,0.73,0.73}{##1}}}
\@namedef{PY@tok@c}{\let\PY@it=\textit\def\PY@tc##1{\textcolor[rgb]{0.25,0.50,0.50}{##1}}}
\@namedef{PY@tok@cp}{\def\PY@tc##1{\textcolor[rgb]{0.74,0.48,0.00}{##1}}}
\@namedef{PY@tok@k}{\let\PY@bf=\textbf\def\PY@tc##1{\textcolor[rgb]{0.00,0.50,0.00}{##1}}}
\@namedef{PY@tok@kp}{\def\PY@tc##1{\textcolor[rgb]{0.00,0.50,0.00}{##1}}}
\@namedef{PY@tok@kt}{\def\PY@tc##1{\textcolor[rgb]{0.69,0.00,0.25}{##1}}}
\@namedef{PY@tok@o}{\def\PY@tc##1{\textcolor[rgb]{0.40,0.40,0.40}{##1}}}
\@namedef{PY@tok@ow}{\let\PY@bf=\textbf\def\PY@tc##1{\textcolor[rgb]{0.67,0.13,1.00}{##1}}}
\@namedef{PY@tok@nb}{\def\PY@tc##1{\textcolor[rgb]{0.00,0.50,0.00}{##1}}}
\@namedef{PY@tok@nf}{\def\PY@tc##1{\textcolor[rgb]{0.00,0.00,1.00}{##1}}}
\@namedef{PY@tok@nc}{\let\PY@bf=\textbf\def\PY@tc##1{\textcolor[rgb]{0.00,0.00,1.00}{##1}}}
\@namedef{PY@tok@nn}{\let\PY@bf=\textbf\def\PY@tc##1{\textcolor[rgb]{0.00,0.00,1.00}{##1}}}
\@namedef{PY@tok@ne}{\let\PY@bf=\textbf\def\PY@tc##1{\textcolor[rgb]{0.82,0.25,0.23}{##1}}}
\@namedef{PY@tok@nv}{\def\PY@tc##1{\textcolor[rgb]{0.10,0.09,0.49}{##1}}}
\@namedef{PY@tok@no}{\def\PY@tc##1{\textcolor[rgb]{0.53,0.00,0.00}{##1}}}
\@namedef{PY@tok@nl}{\def\PY@tc##1{\textcolor[rgb]{0.63,0.63,0.00}{##1}}}
\@namedef{PY@tok@ni}{\let\PY@bf=\textbf\def\PY@tc##1{\textcolor[rgb]{0.60,0.60,0.60}{##1}}}
\@namedef{PY@tok@na}{\def\PY@tc##1{\textcolor[rgb]{0.49,0.56,0.16}{##1}}}
\@namedef{PY@tok@nt}{\let\PY@bf=\textbf\def\PY@tc##1{\textcolor[rgb]{0.00,0.50,0.00}{##1}}}
\@namedef{PY@tok@nd}{\def\PY@tc##1{\textcolor[rgb]{0.67,0.13,1.00}{##1}}}
\@namedef{PY@tok@s}{\def\PY@tc##1{\textcolor[rgb]{0.73,0.13,0.13}{##1}}}
\@namedef{PY@tok@sd}{\let\PY@it=\textit\def\PY@tc##1{\textcolor[rgb]{0.73,0.13,0.13}{##1}}}
\@namedef{PY@tok@si}{\let\PY@bf=\textbf\def\PY@tc##1{\textcolor[rgb]{0.73,0.40,0.53}{##1}}}
\@namedef{PY@tok@se}{\let\PY@bf=\textbf\def\PY@tc##1{\textcolor[rgb]{0.73,0.40,0.13}{##1}}}
\@namedef{PY@tok@sr}{\def\PY@tc##1{\textcolor[rgb]{0.73,0.40,0.53}{##1}}}
\@namedef{PY@tok@ss}{\def\PY@tc##1{\textcolor[rgb]{0.10,0.09,0.49}{##1}}}
\@namedef{PY@tok@sx}{\def\PY@tc##1{\textcolor[rgb]{0.00,0.50,0.00}{##1}}}
\@namedef{PY@tok@m}{\def\PY@tc##1{\textcolor[rgb]{0.40,0.40,0.40}{##1}}}
\@namedef{PY@tok@gh}{\let\PY@bf=\textbf\def\PY@tc##1{\textcolor[rgb]{0.00,0.00,0.50}{##1}}}
\@namedef{PY@tok@gu}{\let\PY@bf=\textbf\def\PY@tc##1{\textcolor[rgb]{0.50,0.00,0.50}{##1}}}
\@namedef{PY@tok@gd}{\def\PY@tc##1{\textcolor[rgb]{0.63,0.00,0.00}{##1}}}
\@namedef{PY@tok@gi}{\def\PY@tc##1{\textcolor[rgb]{0.00,0.63,0.00}{##1}}}
\@namedef{PY@tok@gr}{\def\PY@tc##1{\textcolor[rgb]{1.00,0.00,0.00}{##1}}}
\@namedef{PY@tok@ge}{\let\PY@it=\textit}
\@namedef{PY@tok@gs}{\let\PY@bf=\textbf}
\@namedef{PY@tok@gp}{\let\PY@bf=\textbf\def\PY@tc##1{\textcolor[rgb]{0.00,0.00,0.50}{##1}}}
\@namedef{PY@tok@go}{\def\PY@tc##1{\textcolor[rgb]{0.53,0.53,0.53}{##1}}}
\@namedef{PY@tok@gt}{\def\PY@tc##1{\textcolor[rgb]{0.00,0.27,0.87}{##1}}}
\@namedef{PY@tok@err}{\def\PY@bc##1{{\setlength{\fboxsep}{\string -\fboxrule}\fcolorbox[rgb]{1.00,0.00,0.00}{1,1,1}{\strut ##1}}}}
\@namedef{PY@tok@kc}{\let\PY@bf=\textbf\def\PY@tc##1{\textcolor[rgb]{0.00,0.50,0.00}{##1}}}
\@namedef{PY@tok@kd}{\let\PY@bf=\textbf\def\PY@tc##1{\textcolor[rgb]{0.00,0.50,0.00}{##1}}}
\@namedef{PY@tok@kn}{\let\PY@bf=\textbf\def\PY@tc##1{\textcolor[rgb]{0.00,0.50,0.00}{##1}}}
\@namedef{PY@tok@kr}{\let\PY@bf=\textbf\def\PY@tc##1{\textcolor[rgb]{0.00,0.50,0.00}{##1}}}
\@namedef{PY@tok@bp}{\def\PY@tc##1{\textcolor[rgb]{0.00,0.50,0.00}{##1}}}
\@namedef{PY@tok@fm}{\def\PY@tc##1{\textcolor[rgb]{0.00,0.00,1.00}{##1}}}
\@namedef{PY@tok@vc}{\def\PY@tc##1{\textcolor[rgb]{0.10,0.09,0.49}{##1}}}
\@namedef{PY@tok@vg}{\def\PY@tc##1{\textcolor[rgb]{0.10,0.09,0.49}{##1}}}
\@namedef{PY@tok@vi}{\def\PY@tc##1{\textcolor[rgb]{0.10,0.09,0.49}{##1}}}
\@namedef{PY@tok@vm}{\def\PY@tc##1{\textcolor[rgb]{0.10,0.09,0.49}{##1}}}
\@namedef{PY@tok@sa}{\def\PY@tc##1{\textcolor[rgb]{0.73,0.13,0.13}{##1}}}
\@namedef{PY@tok@sb}{\def\PY@tc##1{\textcolor[rgb]{0.73,0.13,0.13}{##1}}}
\@namedef{PY@tok@sc}{\def\PY@tc##1{\textcolor[rgb]{0.73,0.13,0.13}{##1}}}
\@namedef{PY@tok@dl}{\def\PY@tc##1{\textcolor[rgb]{0.73,0.13,0.13}{##1}}}
\@namedef{PY@tok@s2}{\def\PY@tc##1{\textcolor[rgb]{0.73,0.13,0.13}{##1}}}
\@namedef{PY@tok@sh}{\def\PY@tc##1{\textcolor[rgb]{0.73,0.13,0.13}{##1}}}
\@namedef{PY@tok@s1}{\def\PY@tc##1{\textcolor[rgb]{0.73,0.13,0.13}{##1}}}
\@namedef{PY@tok@mb}{\def\PY@tc##1{\textcolor[rgb]{0.40,0.40,0.40}{##1}}}
\@namedef{PY@tok@mf}{\def\PY@tc##1{\textcolor[rgb]{0.40,0.40,0.40}{##1}}}
\@namedef{PY@tok@mh}{\def\PY@tc##1{\textcolor[rgb]{0.40,0.40,0.40}{##1}}}
\@namedef{PY@tok@mi}{\def\PY@tc##1{\textcolor[rgb]{0.40,0.40,0.40}{##1}}}
\@namedef{PY@tok@il}{\def\PY@tc##1{\textcolor[rgb]{0.40,0.40,0.40}{##1}}}
\@namedef{PY@tok@mo}{\def\PY@tc##1{\textcolor[rgb]{0.40,0.40,0.40}{##1}}}
\@namedef{PY@tok@ch}{\let\PY@it=\textit\def\PY@tc##1{\textcolor[rgb]{0.25,0.50,0.50}{##1}}}
\@namedef{PY@tok@cm}{\let\PY@it=\textit\def\PY@tc##1{\textcolor[rgb]{0.25,0.50,0.50}{##1}}}
\@namedef{PY@tok@cpf}{\let\PY@it=\textit\def\PY@tc##1{\textcolor[rgb]{0.25,0.50,0.50}{##1}}}
\@namedef{PY@tok@c1}{\let\PY@it=\textit\def\PY@tc##1{\textcolor[rgb]{0.25,0.50,0.50}{##1}}}
\@namedef{PY@tok@cs}{\let\PY@it=\textit\def\PY@tc##1{\textcolor[rgb]{0.25,0.50,0.50}{##1}}}

\def\PYZbs{\char`\\}
\def\PYZus{\char`\_}
\def\PYZob{\char`\{}
\def\PYZcb{\char`\}}
\def\PYZca{\char`\^}
\def\PYZam{\char`\&}
\def\PYZlt{\char`\<}
\def\PYZgt{\char`\>}
\def\PYZsh{\char`\#}
\def\PYZpc{\char`\%}
\def\PYZdl{\char`\$}
\def\PYZhy{\char`\-}
\def\PYZsq{\char`\'}
\def\PYZdq{\char`\"}
\def\PYZti{\char`\~}
% for compatibility with earlier versions
\def\PYZat{@}
\def\PYZlb{[}
\def\PYZrb{]}
\makeatother


    % For linebreaks inside Verbatim environment from package fancyvrb. 
    \makeatletter
        \newbox\Wrappedcontinuationbox 
        \newbox\Wrappedvisiblespacebox 
        \newcommand*\Wrappedvisiblespace {\textcolor{red}{\textvisiblespace}} 
        \newcommand*\Wrappedcontinuationsymbol {\textcolor{red}{\llap{\tiny$\m@th\hookrightarrow$}}} 
        \newcommand*\Wrappedcontinuationindent {3ex } 
        \newcommand*\Wrappedafterbreak {\kern\Wrappedcontinuationindent\copy\Wrappedcontinuationbox} 
        % Take advantage of the already applied Pygments mark-up to insert 
        % potential linebreaks for TeX processing. 
        %        {, <, #, %, $, ' and ": go to next line. 
        %        _, }, ^, &, >, - and ~: stay at end of broken line. 
        % Use of \textquotesingle for straight quote. 
        \newcommand*\Wrappedbreaksatspecials {% 
            \def\PYGZus{\discretionary{\char`\_}{\Wrappedafterbreak}{\char`\_}}% 
            \def\PYGZob{\discretionary{}{\Wrappedafterbreak\char`\{}{\char`\{}}% 
            \def\PYGZcb{\discretionary{\char`\}}{\Wrappedafterbreak}{\char`\}}}% 
            \def\PYGZca{\discretionary{\char`\^}{\Wrappedafterbreak}{\char`\^}}% 
            \def\PYGZam{\discretionary{\char`\&}{\Wrappedafterbreak}{\char`\&}}% 
            \def\PYGZlt{\discretionary{}{\Wrappedafterbreak\char`\<}{\char`\<}}% 
            \def\PYGZgt{\discretionary{\char`\>}{\Wrappedafterbreak}{\char`\>}}% 
            \def\PYGZsh{\discretionary{}{\Wrappedafterbreak\char`\#}{\char`\#}}% 
            \def\PYGZpc{\discretionary{}{\Wrappedafterbreak\char`\%}{\char`\%}}% 
            \def\PYGZdl{\discretionary{}{\Wrappedafterbreak\char`\$}{\char`\$}}% 
            \def\PYGZhy{\discretionary{\char`\-}{\Wrappedafterbreak}{\char`\-}}% 
            \def\PYGZsq{\discretionary{}{\Wrappedafterbreak\textquotesingle}{\textquotesingle}}% 
            \def\PYGZdq{\discretionary{}{\Wrappedafterbreak\char`\"}{\char`\"}}% 
            \def\PYGZti{\discretionary{\char`\~}{\Wrappedafterbreak}{\char`\~}}% 
        } 
        % Some characters . , ; ? ! / are not pygmentized. 
        % This macro makes them "active" and they will insert potential linebreaks 
        \newcommand*\Wrappedbreaksatpunct {% 
            \lccode`\~`\.\lowercase{\def~}{\discretionary{\hbox{\char`\.}}{\Wrappedafterbreak}{\hbox{\char`\.}}}% 
            \lccode`\~`\,\lowercase{\def~}{\discretionary{\hbox{\char`\,}}{\Wrappedafterbreak}{\hbox{\char`\,}}}% 
            \lccode`\~`\;\lowercase{\def~}{\discretionary{\hbox{\char`\;}}{\Wrappedafterbreak}{\hbox{\char`\;}}}% 
            \lccode`\~`\:\lowercase{\def~}{\discretionary{\hbox{\char`\:}}{\Wrappedafterbreak}{\hbox{\char`\:}}}% 
            \lccode`\~`\?\lowercase{\def~}{\discretionary{\hbox{\char`\?}}{\Wrappedafterbreak}{\hbox{\char`\?}}}% 
            \lccode`\~`\!\lowercase{\def~}{\discretionary{\hbox{\char`\!}}{\Wrappedafterbreak}{\hbox{\char`\!}}}% 
            \lccode`\~`\/\lowercase{\def~}{\discretionary{\hbox{\char`\/}}{\Wrappedafterbreak}{\hbox{\char`\/}}}% 
            \catcode`\.\active
            \catcode`\,\active 
            \catcode`\;\active
            \catcode`\:\active
            \catcode`\?\active
            \catcode`\!\active
            \catcode`\/\active 
            \lccode`\~`\~ 	
        }
    \makeatother

    \let\OriginalVerbatim=\Verbatim
    \makeatletter
    \renewcommand{\Verbatim}[1][1]{%
        %\parskip\z@skip
        \sbox\Wrappedcontinuationbox {\Wrappedcontinuationsymbol}%
        \sbox\Wrappedvisiblespacebox {\FV@SetupFont\Wrappedvisiblespace}%
        \def\FancyVerbFormatLine ##1{\hsize\linewidth
            \vtop{\raggedright\hyphenpenalty\z@\exhyphenpenalty\z@
                \doublehyphendemerits\z@\finalhyphendemerits\z@
                \strut ##1\strut}%
        }%
        % If the linebreak is at a space, the latter will be displayed as visible
        % space at end of first line, and a continuation symbol starts next line.
        % Stretch/shrink are however usually zero for typewriter font.
        \def\FV@Space {%
            \nobreak\hskip\z@ plus\fontdimen3\font minus\fontdimen4\font
            \discretionary{\copy\Wrappedvisiblespacebox}{\Wrappedafterbreak}
            {\kern\fontdimen2\font}%
        }%
        
        % Allow breaks at special characters using \PYG... macros.
        \Wrappedbreaksatspecials
        % Breaks at punctuation characters . , ; ? ! and / need catcode=\active 	
        \OriginalVerbatim[#1,codes*=\Wrappedbreaksatpunct]%
    }
    \makeatother

    % Exact colors from NB
    \definecolor{incolor}{HTML}{303F9F}
    \definecolor{outcolor}{HTML}{D84315}
    \definecolor{cellborder}{HTML}{CFCFCF}
    \definecolor{cellbackground}{HTML}{F7F7F7}
    
    % prompt
    \makeatletter
    \newcommand{\boxspacing}{\kern\kvtcb@left@rule\kern\kvtcb@boxsep}
    \makeatother
    \newcommand{\prompt}[4]{
        {\ttfamily\llap{{\color{#2}[#3]:\hspace{3pt}#4}}\vspace{-\baselineskip}}
    }
    

    
    % Prevent overflowing lines due to hard-to-break entities
    \sloppy 
    % Setup hyperref package
    \hypersetup{
      breaklinks=true,  % so long urls are correctly broken across lines
      colorlinks=true,
      urlcolor=urlcolor,
      linkcolor=linkcolor,
      citecolor=citecolor,
      }
    % Slightly bigger margins than the latex defaults
    
    \geometry{verbose,tmargin=1in,bmargin=1in,lmargin=1in,rmargin=1in}
    
    

\begin{document}
    
    \pagenumbering{gobble}

% Title page
\begin{titlepage}
  \begin{minipage}{\textwidth}
    \centering
    \includegraphics[scale=0.25]{img/ub-logo}\\[2cm]
    
    \textsc{\Large \subject\\[0.5cm]}
    \textsc{\uppercase\expandafter{\masters}}\\[1.5cm]
    
    \noindent\rule[-1ex]{\textwidth}{1pt}\\[1.5ex]
    \textsc{{\Huge \titulo\\[0.5ex]}}
    \textsc{{\Large \subtitulo\\}}
    \noindent\rule[-1ex]{\textwidth}{2pt}\\[3.5ex]
  \end{minipage}
  
  \vspace{2cm}
  
  \begin{minipage}{\textwidth}
    \centering
    
    \includegraphics[scale=0.4]{img/ub-ds-logo}
    \vspace{2cm}
    
    \textbf{Author}\\ {\autor{}}\\[2.5ex]
    \textsc{Faculty of Mathematics and Computer Science}\\
    \vspace{1em}
    \textsc{Academic year 2021-2022}
  \end{minipage}
\end{titlepage}

\pagenumbering{arabic}
\setlength{\parskip}{1em}

    
    

    
    \hypertarget{problem-description}{%
\section{Problem description}}

\textbf{Use a genetic algorithm to solve the example of page 84 with
population size \(M = 50\):}

\[ \max f(x,y) = 21.5 + x \sin(4\pi x) + y\sin(20\pi y) \]

\textbf{with}

\[(x, y) \in [-3.0, 12.11] \times [4.5, 5.8]\]

\hypertarget{solution}{%
\section{Solution}}

To solve this problem we have implemented a genetic algorithm that uses
real-valued representation instead of the binary representation because
it feels more natural to the given problem. This simplifies a lot the
operations that have to be carried out and is more understandable and
natural. However, there are some considerations that have to be taken
into account.

To begin with, whenever a new chromosome is generated or mutated, we
have to make sure that the chromosome is still a feasible solution to
the problem. Therefore, we have implemented a method that makes sure
that the genes of the chromosomes are still in the feasible region by
clipping their values. This operation is applied after every crossover
or mutation.

Secondly, we cannot use the same crossover as in the binary case because
we cannot interchange as easily as before parts of two chromosomes.
Instead, we are going to use the \textbf{BLX-\(\alpha\)} crossover.
Given two chromosomes, \(C_1 = (c_1^1, \dots, c_n^1)\) and
\(C_2 = (c_1^2, \dots, c_n^2)\), it generates two offsprings
\(O_1 = (o_1^1, \dots, o_n^1)\) and \(O_2 = (o_1^2, \dots, o_n^2)\)
where each \(o_i^k\) is randomly (uniformly) chosen from the interval
\([c_{min} - I\alpha, c_{max} + I\alpha]\), where
\(c_{max} = \max(c_i^1, c_i^2)\), \(c_{min} = \min(c_i^1, c_i^2)\) and
\(I = c_{max} - c_{min}\). In other words, for each pair of genes of the
parents we generate a new random uniform value in the interval between
them and also in the extrema near them. This way, we can explore the
surroundings of the current solutions by choosing values outside the
extrema of the interval and we can exploit the better solutions by
choosing values in the interval.

Finally, we cannot apply the same mutation as in the binary case. For
real numbers, we can generate a random value from a normal distribution
with mean \(\mu\) and standard deviation \(\sigma\) and add it to the
selected gene.

Let's describe now how the algorithm works. We generate a population of
size \(M = 50\) and we evaluate them. To make things easier, we sort the
population based on the corresponding fitness values from best to worst.
Then we perform the optimization process. In each iteration, we select
the chromosomes that will make it into the next generation. To do this,
we select random pairs of chromosomes from the population, and for each
one of them we perform a \textbf{binary tournament}, selecting the
chromosome with the best fitness value. We repeat this process until we
have a new population with \(M\) chromosomes (they can be repeated).

Having selected the members of the new population, we perform the
crossover. To do so, we generate \(M\) random values from a \([0, 1)\)
uniform distribution. Then we check the value of each random value. If
it's smaller than the cross rate, then the corresponding chromosome is
selected for mating. If we get an even number of chromosomes, we just
remove the last one. We generate random couples and we apply the
\textbf{BLX-\(\alpha\)} crossover.

After that, we try to mutate the new population. To do this, we generate
a random value for each gene of each chromosome of the population. If
it's smaller than the mutation rate then the previously defined mutation
is applied to the gene. After we do that, we evaluate the new population
and sort it according to its fitness values.

Finally, we compare the previous population with the new one and we
apply an \textbf{elitism mnechanism}, which will allow us to keep the
best solution found so far in the population. To implement, we compare
the fitness values of the best chromosome from the previous generation
with the best from the current. If the chromosome from the previous one
is better, then we substitute the worst from the current and we sort
again the population. Then, we repeat the process.

Let's see now an implementation of this algorithm:

    \begin{tcolorbox}[breakable, size=fbox, boxrule=1pt, pad at break*=1mm,colback=cellbackground, colframe=cellborder]
\prompt{In}{incolor}{1}{\boxspacing}
\begin{Verbatim}[commandchars=\\\{\}]
\PY{k+kn}{import} \PY{n+nn}{numpy} \PY{k}{as} \PY{n+nn}{np}
\PY{k+kn}{import} \PY{n+nn}{matplotlib}\PY{n+nn}{.}\PY{n+nn}{pyplot} \PY{k}{as} \PY{n+nn}{plt}
\end{Verbatim}
\end{tcolorbox}

    \begin{tcolorbox}[breakable, size=fbox, boxrule=1pt, pad at break*=1mm,colback=cellbackground, colframe=cellborder]
\prompt{In}{incolor}{2}{\boxspacing}
\begin{Verbatim}[commandchars=\\\{\}]
\PY{k}{class} \PY{n+nc}{GeneticAlgorithm}\PY{p}{:}
    \PY{k}{def} \PY{n+nf+fm}{\PYZus{}\PYZus{}init\PYZus{}\PYZus{}}\PY{p}{(}\PY{n+nb+bp}{self}\PY{p}{,} \PY{n}{pop\PYZus{}size}\PY{p}{,} \PY{n}{x\PYZus{}range}\PY{p}{,} \PY{n}{y\PYZus{}range}\PY{p}{,} \PY{n}{max\PYZus{}iter}\PY{o}{=}\PY{l+m+mi}{10000}\PY{p}{,} \PY{n}{cross\PYZus{}rate}\PY{o}{=}\PY{l+m+mf}{0.7}\PY{p}{,} \PY{n}{mutation\PYZus{}rate}\PY{o}{=}\PY{l+m+mf}{0.001}\PY{p}{)}\PY{p}{:}
        \PY{n+nb+bp}{self}\PY{o}{.}\PY{n}{pop\PYZus{}size} \PY{o}{=} \PY{n}{pop\PYZus{}size}
        \PY{n+nb+bp}{self}\PY{o}{.}\PY{n}{x\PYZus{}range} \PY{o}{=} \PY{n}{x\PYZus{}range}
        \PY{n+nb+bp}{self}\PY{o}{.}\PY{n}{y\PYZus{}range} \PY{o}{=} \PY{n}{y\PYZus{}range}
        \PY{n+nb+bp}{self}\PY{o}{.}\PY{n}{max\PYZus{}iter} \PY{o}{=} \PY{n}{max\PYZus{}iter}
        \PY{n+nb+bp}{self}\PY{o}{.}\PY{n}{cross\PYZus{}rate} \PY{o}{=} \PY{n}{cross\PYZus{}rate}
        \PY{n+nb+bp}{self}\PY{o}{.}\PY{n}{mutation\PYZus{}rate} \PY{o}{=} \PY{n}{mutation\PYZus{}rate}
    
    
    \PY{k}{def} \PY{n+nf}{\PYZus{}initialize\PYZus{}population}\PY{p}{(}\PY{n+nb+bp}{self}\PY{p}{)}\PY{p}{:}
        \PY{l+s+sd}{\PYZsq{}\PYZsq{}\PYZsq{}}
\PY{l+s+sd}{        Method to initialize a population of chromosomes}
\PY{l+s+sd}{        \PYZsq{}\PYZsq{}\PYZsq{}}
        \PY{n}{population} \PY{o}{=} \PY{n}{np}\PY{o}{.}\PY{n}{random}\PY{o}{.}\PY{n}{uniform}\PY{p}{(}\PY{n}{size}\PY{o}{=}\PY{p}{(}\PY{n+nb+bp}{self}\PY{o}{.}\PY{n}{pop\PYZus{}size}\PY{p}{,} \PY{l+m+mi}{2}\PY{p}{)}\PY{p}{)}
        
        \PY{n}{x\PYZus{}interval} \PY{o}{=} \PY{n+nb+bp}{self}\PY{o}{.}\PY{n}{x\PYZus{}range}\PY{p}{[}\PY{l+m+mi}{1}\PY{p}{]} \PY{o}{\PYZhy{}} \PY{n+nb+bp}{self}\PY{o}{.}\PY{n}{x\PYZus{}range}\PY{p}{[}\PY{l+m+mi}{0}\PY{p}{]}
        \PY{n}{y\PYZus{}interval} \PY{o}{=} \PY{n+nb+bp}{self}\PY{o}{.}\PY{n}{y\PYZus{}range}\PY{p}{[}\PY{l+m+mi}{1}\PY{p}{]} \PY{o}{\PYZhy{}} \PY{n+nb+bp}{self}\PY{o}{.}\PY{n}{y\PYZus{}range}\PY{p}{[}\PY{l+m+mi}{0}\PY{p}{]}
        
        \PY{c+c1}{\PYZsh{} Map [0, 1] values to corresponding ranges}
        \PY{n}{population}\PY{p}{[}\PY{p}{:}\PY{p}{,} \PY{l+m+mi}{0}\PY{p}{]} \PY{o}{=} \PY{n}{population}\PY{p}{[}\PY{p}{:}\PY{p}{,} \PY{l+m+mi}{0}\PY{p}{]} \PY{o}{*} \PY{n}{x\PYZus{}interval} \PY{o}{+} \PY{n+nb+bp}{self}\PY{o}{.}\PY{n}{x\PYZus{}range}\PY{p}{[}\PY{l+m+mi}{0}\PY{p}{]}
        \PY{n}{population}\PY{p}{[}\PY{p}{:}\PY{p}{,} \PY{l+m+mi}{1}\PY{p}{]} \PY{o}{=} \PY{n}{population}\PY{p}{[}\PY{p}{:}\PY{p}{,} \PY{l+m+mi}{1}\PY{p}{]} \PY{o}{*} \PY{n}{y\PYZus{}interval} \PY{o}{+} \PY{n+nb+bp}{self}\PY{o}{.}\PY{n}{y\PYZus{}range}\PY{p}{[}\PY{l+m+mi}{0}\PY{p}{]}
        
        \PY{k}{return} \PY{n}{population}
    
    
    \PY{k}{def} \PY{n+nf}{\PYZus{}evaluate\PYZus{}population}\PY{p}{(}\PY{n+nb+bp}{self}\PY{p}{,} \PY{n}{population}\PY{p}{)}\PY{p}{:}
        \PY{l+s+sd}{\PYZsq{}\PYZsq{}\PYZsq{}}
\PY{l+s+sd}{        Method to evaluate the population of chromosomes using the fitness function.}
\PY{l+s+sd}{        \PYZsq{}\PYZsq{}\PYZsq{}}
        \PY{n}{x\PYZus{}values} \PY{o}{=} \PY{n}{population}\PY{p}{[}\PY{p}{:}\PY{p}{,} \PY{l+m+mi}{0}\PY{p}{]}
        \PY{n}{y\PYZus{}values} \PY{o}{=} \PY{n}{population}\PY{p}{[}\PY{p}{:}\PY{p}{,} \PY{l+m+mi}{1}\PY{p}{]}
        
        \PY{k}{return} \PY{l+m+mf}{21.5} \PY{o}{+} \PY{n}{x\PYZus{}values} \PY{o}{*} \PY{n}{np}\PY{o}{.}\PY{n}{sin}\PY{p}{(}\PY{l+m+mi}{4} \PY{o}{*} \PY{n}{np}\PY{o}{.}\PY{n}{pi} \PY{o}{*} \PY{n}{x\PYZus{}values}\PY{p}{)} \PY{o}{+} \PY{n}{y\PYZus{}values} \PY{o}{*} \PY{n}{np}\PY{o}{.}\PY{n}{sin}\PY{p}{(}\PY{l+m+mi}{20} \PY{o}{*} \PY{n}{np}\PY{o}{.}\PY{n}{pi} \PY{o}{*} \PY{n}{y\PYZus{}values}\PY{p}{)}
    
    
    \PY{k}{def} \PY{n+nf}{\PYZus{}sort\PYZus{}population\PYZus{}fitness}\PY{p}{(}\PY{n+nb+bp}{self}\PY{p}{,} \PY{n}{population}\PY{p}{,} \PY{n}{fitness}\PY{p}{)}\PY{p}{:}
        \PY{l+s+sd}{\PYZsq{}\PYZsq{}\PYZsq{}}
\PY{l+s+sd}{        Method used to sort the population according to their fitness values.}
\PY{l+s+sd}{        \PYZsq{}\PYZsq{}\PYZsq{}}
        \PY{n}{sort\PYZus{}idx} \PY{o}{=} \PY{n}{np}\PY{o}{.}\PY{n}{argsort}\PY{p}{(}\PY{n}{fitness}\PY{p}{)}\PY{p}{[}\PY{p}{:}\PY{p}{:}\PY{o}{\PYZhy{}}\PY{l+m+mi}{1}\PY{p}{]}
        
        \PY{k}{return} \PY{n}{population}\PY{p}{[}\PY{n}{sort\PYZus{}idx}\PY{p}{]}\PY{p}{,} \PY{n}{fitness}\PY{p}{[}\PY{n}{sort\PYZus{}idx}\PY{p}{]}
    
    
    \PY{k}{def} \PY{n+nf}{\PYZus{}selection\PYZus{}schema}\PY{p}{(}\PY{n+nb+bp}{self}\PY{p}{,} \PY{n}{population}\PY{p}{,} \PY{n}{fitness}\PY{p}{)}\PY{p}{:}
        \PY{l+s+sd}{\PYZsq{}\PYZsq{}\PYZsq{}}
\PY{l+s+sd}{        Method that implements the selection schema. It\PYZsq{}s based on the}
\PY{l+s+sd}{        binary tournament, in which we select pairs of chromosomes that}
\PY{l+s+sd}{        compete to see who has the best fitness value. The tournament is}
\PY{l+s+sd}{        applied until we have a new population of the same size as the}
\PY{l+s+sd}{        previous one.}
\PY{l+s+sd}{        \PYZsq{}\PYZsq{}\PYZsq{}}
        \PY{c+c1}{\PYZsh{} The selection schema is based on the binary torunament}
        \PY{n}{new\PYZus{}population} \PY{o}{=} \PY{p}{[}\PY{p}{]}
        
        \PY{k}{for} \PY{n}{\PYZus{}} \PY{o+ow}{in} \PY{n+nb}{range}\PY{p}{(}\PY{n+nb+bp}{self}\PY{o}{.}\PY{n}{pop\PYZus{}size}\PY{p}{)}\PY{p}{:}
            \PY{n}{idx\PYZus{}1}\PY{p}{,} \PY{n}{idx\PYZus{}2} \PY{o}{=} \PY{n}{np}\PY{o}{.}\PY{n}{random}\PY{o}{.}\PY{n}{choice}\PY{p}{(}\PY{n+nb+bp}{self}\PY{o}{.}\PY{n}{pop\PYZus{}size}\PY{p}{,} \PY{l+m+mi}{2}\PY{p}{)}
            \PY{n}{new\PYZus{}population}\PY{o}{.}\PY{n}{append}\PY{p}{(}\PY{n}{population}\PY{p}{[}\PY{n}{idx\PYZus{}1}\PY{p}{]} \PY{k}{if} \PY{n}{fitness}\PY{p}{[}\PY{n}{idx\PYZus{}1}\PY{p}{]} \PY{o}{\PYZgt{}} \PY{n}{fitness}\PY{p}{[}\PY{n}{idx\PYZus{}2}\PY{p}{]} \PY{k}{else} \PY{n}{population}\PY{p}{[}\PY{n}{idx\PYZus{}2}\PY{p}{]}\PY{p}{)}
        
        \PY{n}{new\PYZus{}population} \PY{o}{=} \PY{n}{np}\PY{o}{.}\PY{n}{array}\PY{p}{(}\PY{n}{new\PYZus{}population}\PY{p}{)}
        
        \PY{k}{return} \PY{n}{new\PYZus{}population}
    
    
    \PY{k}{def} \PY{n+nf}{\PYZus{}blx\PYZus{}alpha\PYZus{}crossover}\PY{p}{(}\PY{n+nb+bp}{self}\PY{p}{,} \PY{n}{parents}\PY{p}{,} \PY{n}{alpha}\PY{o}{=}\PY{l+m+mf}{0.2}\PY{p}{)}\PY{p}{:}
        \PY{l+s+sd}{\PYZsq{}\PYZsq{}\PYZsq{}}
\PY{l+s+sd}{        Method that implements the BLX\PYZhy{}\PYZbs{}alpha crossover operator. This}
\PY{l+s+sd}{        is one of the crossover operators used for genetic algorithms}
\PY{l+s+sd}{        that use the real\PYZhy{}valued repesentation.}
\PY{l+s+sd}{        \PYZsq{}\PYZsq{}\PYZsq{}}
        \PY{c+c1}{\PYZsh{} Compute min and max values column\PYZhy{}wise}
        \PY{n}{c\PYZus{}min} \PY{o}{=} \PY{n}{np}\PY{o}{.}\PY{n}{min}\PY{p}{(}\PY{n}{parents}\PY{p}{,} \PY{n}{axis}\PY{o}{=}\PY{l+m+mi}{0}\PY{p}{)}
        \PY{n}{c\PYZus{}max} \PY{o}{=} \PY{n}{np}\PY{o}{.}\PY{n}{max}\PY{p}{(}\PY{n}{parents}\PY{p}{,} \PY{n}{axis}\PY{o}{=}\PY{l+m+mi}{0}\PY{p}{)}

        \PY{c+c1}{\PYZsh{} Compute interval}
        \PY{n}{i} \PY{o}{=} \PY{n}{c\PYZus{}max} \PY{o}{\PYZhy{}} \PY{n}{c\PYZus{}min}
        
        \PY{c+c1}{\PYZsh{} Generate children}
        \PY{n}{child\PYZus{}1} \PY{o}{=} \PY{n}{np}\PY{o}{.}\PY{n}{random}\PY{o}{.}\PY{n}{uniform}\PY{p}{(}\PY{n}{c\PYZus{}min} \PY{o}{\PYZhy{}} \PY{n}{i} \PY{o}{*} \PY{n}{alpha}\PY{p}{,} \PY{n}{c\PYZus{}max} \PY{o}{+} \PY{n}{i} \PY{o}{*} \PY{n}{alpha}\PY{p}{)}
        \PY{n}{child\PYZus{}2} \PY{o}{=} \PY{n}{np}\PY{o}{.}\PY{n}{random}\PY{o}{.}\PY{n}{uniform}\PY{p}{(}\PY{n}{c\PYZus{}min} \PY{o}{\PYZhy{}} \PY{n}{i} \PY{o}{*} \PY{n}{alpha}\PY{p}{,} \PY{n}{c\PYZus{}max} \PY{o}{+} \PY{n}{i} \PY{o}{*} \PY{n}{alpha}\PY{p}{)}
        
        \PY{n}{children} \PY{o}{=} \PY{n}{np}\PY{o}{.}\PY{n}{vstack}\PY{p}{(}\PY{p}{(}\PY{n}{child\PYZus{}1}\PY{p}{,} \PY{n}{child\PYZus{}2}\PY{p}{)}\PY{p}{)}
        \PY{n}{children} \PY{o}{=} \PY{n+nb+bp}{self}\PY{o}{.}\PY{n}{\PYZus{}clip\PYZus{}values}\PY{p}{(}\PY{n}{children}\PY{p}{)}
        
        \PY{k}{return} \PY{n}{children}
        
    
    \PY{k}{def} \PY{n+nf}{\PYZus{}cross\PYZus{}schema}\PY{p}{(}\PY{n+nb+bp}{self}\PY{p}{,} \PY{n}{population}\PY{p}{)}\PY{p}{:}
        \PY{l+s+sd}{\PYZsq{}\PYZsq{}\PYZsq{}}
\PY{l+s+sd}{        Method that implements the cross schema. For every chromosome}
\PY{l+s+sd}{        in the population, a random number is generated and if it\PYZsq{}s below}
\PY{l+s+sd}{        the threshold, the chromosome is selected for breeding. Then, random}
\PY{l+s+sd}{        couples are formed and children are formed. These children replace}
\PY{l+s+sd}{        their parents.}
\PY{l+s+sd}{        \PYZsq{}\PYZsq{}\PYZsq{}}
        \PY{n}{prob\PYZus{}cross} \PY{o}{=} \PY{n}{np}\PY{o}{.}\PY{n}{random}\PY{o}{.}\PY{n}{uniform}\PY{p}{(}\PY{n}{size}\PY{o}{=}\PY{n+nb+bp}{self}\PY{o}{.}\PY{n}{pop\PYZus{}size}\PY{p}{)}
        \PY{n}{cross\PYZus{}idx} \PY{o}{=} \PY{n}{np}\PY{o}{.}\PY{n}{where}\PY{p}{(}\PY{n}{prob\PYZus{}cross} \PY{o}{\PYZlt{}} \PY{n+nb+bp}{self}\PY{o}{.}\PY{n}{cross\PYZus{}rate}\PY{p}{)}\PY{p}{[}\PY{l+m+mi}{0}\PY{p}{]}
        
        \PY{c+c1}{\PYZsh{} If length is even, remove last element}
        \PY{k}{if} \PY{n+nb}{len}\PY{p}{(}\PY{n}{cross\PYZus{}idx}\PY{p}{)} \PY{o}{\PYZpc{}} \PY{l+m+mi}{2} \PY{o}{!=} \PY{l+m+mi}{0}\PY{p}{:}
            \PY{n}{cross\PYZus{}idx} \PY{o}{=} \PY{n}{cross\PYZus{}idx}\PY{p}{[}\PY{p}{:}\PY{o}{\PYZhy{}}\PY{l+m+mi}{1}\PY{p}{]}
        
        \PY{n}{cross\PYZus{}idx} \PY{o}{=} \PY{n}{np}\PY{o}{.}\PY{n}{random}\PY{o}{.}\PY{n}{permutation}\PY{p}{(}\PY{n}{cross\PYZus{}idx}\PY{p}{)}
        \PY{n}{cross\PYZus{}idx} \PY{o}{=} \PY{n}{cross\PYZus{}idx}\PY{o}{.}\PY{n}{reshape}\PY{p}{(}\PY{o}{\PYZhy{}}\PY{l+m+mi}{1}\PY{p}{,} \PY{l+m+mi}{2}\PY{p}{)}
        
        \PY{k}{for} \PY{n}{couple\PYZus{}idx} \PY{o+ow}{in} \PY{n}{cross\PYZus{}idx}\PY{p}{:}
            \PY{n}{parents} \PY{o}{=} \PY{n}{population}\PY{p}{[}\PY{n}{couple\PYZus{}idx}\PY{p}{]}
            \PY{n}{children} \PY{o}{=} \PY{n+nb+bp}{self}\PY{o}{.}\PY{n}{\PYZus{}blx\PYZus{}alpha\PYZus{}crossover}\PY{p}{(}\PY{n}{parents}\PY{p}{)}
            
            \PY{n}{population}\PY{p}{[}\PY{n}{couple\PYZus{}idx}\PY{p}{]} \PY{o}{=} \PY{n}{children}
            
        \PY{k}{return} \PY{n}{population}
    
    
    \PY{k}{def} \PY{n+nf}{\PYZus{}mutation}\PY{p}{(}\PY{n+nb+bp}{self}\PY{p}{,} \PY{n}{population}\PY{p}{,} \PY{n}{mean}\PY{o}{=}\PY{l+m+mf}{0.}\PY{p}{,} \PY{n}{sigma}\PY{o}{=}\PY{l+m+mf}{0.7}\PY{p}{)}\PY{p}{:}
        \PY{l+s+sd}{\PYZsq{}\PYZsq{}\PYZsq{}}
\PY{l+s+sd}{        Method that implements the mutation operator. It modifies a gene by}
\PY{l+s+sd}{        adding a random value generated from a normal distribution with mean = 0}
\PY{l+s+sd}{        and std=0.7.}
\PY{l+s+sd}{        \PYZsq{}\PYZsq{}\PYZsq{}}
        \PY{n}{mutation\PYZus{}prob} \PY{o}{=} \PY{n}{np}\PY{o}{.}\PY{n}{random}\PY{o}{.}\PY{n}{uniform}\PY{p}{(}\PY{n}{size}\PY{o}{=}\PY{n}{population}\PY{o}{.}\PY{n}{shape}\PY{p}{)}
        \PY{n}{mutation\PYZus{}idx} \PY{o}{=} \PY{n}{np}\PY{o}{.}\PY{n}{where}\PY{p}{(}\PY{n}{mutation\PYZus{}prob} \PY{o}{\PYZlt{}} \PY{n+nb+bp}{self}\PY{o}{.}\PY{n}{mutation\PYZus{}rate}\PY{p}{)}
        
        \PY{n}{population}\PY{p}{[}\PY{n}{mutation\PYZus{}idx}\PY{p}{]} \PY{o}{+}\PY{o}{=} \PY{n}{np}\PY{o}{.}\PY{n}{random}\PY{o}{.}\PY{n}{normal}\PY{p}{(}\PY{n}{mean}\PY{p}{,} \PY{n}{sigma}\PY{p}{,} \PY{n}{size}\PY{o}{=}\PY{n}{mutation\PYZus{}idx}\PY{p}{[}\PY{l+m+mi}{0}\PY{p}{]}\PY{o}{.}\PY{n}{shape}\PY{p}{)}
        \PY{n}{population} \PY{o}{=} \PY{n+nb+bp}{self}\PY{o}{.}\PY{n}{\PYZus{}clip\PYZus{}values}\PY{p}{(}\PY{n}{population}\PY{p}{)}
        
        \PY{k}{return} \PY{n}{population}
    
    
    \PY{k}{def} \PY{n+nf}{\PYZus{}clip\PYZus{}values}\PY{p}{(}\PY{n+nb+bp}{self}\PY{p}{,} \PY{n}{population}\PY{p}{)}\PY{p}{:}
        \PY{l+s+sd}{\PYZsq{}\PYZsq{}\PYZsq{}}
\PY{l+s+sd}{        Method used to make sure that the solutions to the problem are still}
\PY{l+s+sd}{        feasible.}
\PY{l+s+sd}{        \PYZsq{}\PYZsq{}\PYZsq{}}
        \PY{n}{population}\PY{p}{[}\PY{p}{:}\PY{p}{,} \PY{l+m+mi}{0}\PY{p}{]} \PY{o}{=} \PY{n}{np}\PY{o}{.}\PY{n}{clip}\PY{p}{(}\PY{n}{population}\PY{p}{[}\PY{p}{:}\PY{p}{,} \PY{l+m+mi}{0}\PY{p}{]}\PY{p}{,} \PY{o}{*}\PY{n+nb+bp}{self}\PY{o}{.}\PY{n}{x\PYZus{}range}\PY{p}{)}
        \PY{n}{population}\PY{p}{[}\PY{p}{:}\PY{p}{,} \PY{l+m+mi}{1}\PY{p}{]} \PY{o}{=} \PY{n}{np}\PY{o}{.}\PY{n}{clip}\PY{p}{(}\PY{n}{population}\PY{p}{[}\PY{p}{:}\PY{p}{,} \PY{l+m+mi}{1}\PY{p}{]}\PY{p}{,} \PY{o}{*}\PY{n+nb+bp}{self}\PY{o}{.}\PY{n}{y\PYZus{}range}\PY{p}{)}
        
        \PY{k}{return} \PY{n}{population}
    
    
    \PY{k}{def} \PY{n+nf}{\PYZus{}elitism}\PY{p}{(}\PY{n+nb+bp}{self}\PY{p}{,} \PY{n}{old\PYZus{}population}\PY{p}{,} \PY{n}{old\PYZus{}fitness}\PY{p}{,} \PY{n}{new\PYZus{}population}\PY{p}{,} \PY{n}{new\PYZus{}fitness}\PY{p}{)}\PY{p}{:}
        \PY{l+s+sd}{\PYZsq{}\PYZsq{}\PYZsq{}}
\PY{l+s+sd}{        Method used to keep the best solution until now. The worst one is}
\PY{l+s+sd}{        discarded if the previous best solution has a better fitness}
\PY{l+s+sd}{        value than the best solution of the current population.}
\PY{l+s+sd}{        \PYZsq{}\PYZsq{}\PYZsq{}}
        \PY{k}{if} \PY{n}{old\PYZus{}fitness}\PY{p}{[}\PY{l+m+mi}{0}\PY{p}{]} \PY{o}{\PYZgt{}} \PY{n}{new\PYZus{}fitness}\PY{p}{[}\PY{l+m+mi}{0}\PY{p}{]}\PY{p}{:}
            \PY{n}{new\PYZus{}population}\PY{p}{[}\PY{o}{\PYZhy{}}\PY{l+m+mi}{1}\PY{p}{]} \PY{o}{=} \PY{n}{old\PYZus{}population}\PY{p}{[}\PY{l+m+mi}{0}\PY{p}{]}
            \PY{n}{new\PYZus{}fitness}\PY{p}{[}\PY{o}{\PYZhy{}}\PY{l+m+mi}{1}\PY{p}{]} \PY{o}{=} \PY{n}{old\PYZus{}fitness}\PY{p}{[}\PY{l+m+mi}{0}\PY{p}{]}
            
            \PY{n}{new\PYZus{}population}\PY{p}{,} \PY{n}{new\PYZus{}fitness} \PY{o}{=} \PY{n+nb+bp}{self}\PY{o}{.}\PY{n}{\PYZus{}sort\PYZus{}population\PYZus{}fitness}\PY{p}{(}\PY{n}{new\PYZus{}population}\PY{p}{,} \PY{n}{new\PYZus{}fitness}\PY{p}{)}
        
        \PY{k}{return} \PY{n}{new\PYZus{}population}\PY{p}{,} \PY{n}{new\PYZus{}fitness}
    
        
    \PY{k}{def} \PY{n+nf}{train\PYZus{}predict}\PY{p}{(}\PY{n+nb+bp}{self}\PY{p}{,} \PY{n}{verbose}\PY{o}{=}\PY{k+kc}{False}\PY{p}{)}\PY{p}{:}
        \PY{l+s+sd}{\PYZsq{}\PYZsq{}\PYZsq{}}
\PY{l+s+sd}{        Method that trains a population of chromosomes a given number of iterations}
\PY{l+s+sd}{        and returns the best solution.}
\PY{l+s+sd}{        \PYZsq{}\PYZsq{}\PYZsq{}}
        \PY{n}{population} \PY{o}{=} \PY{n+nb+bp}{self}\PY{o}{.}\PY{n}{\PYZus{}initialize\PYZus{}population}\PY{p}{(}\PY{p}{)}
        \PY{n}{fitness} \PY{o}{=} \PY{n+nb+bp}{self}\PY{o}{.}\PY{n}{\PYZus{}evaluate\PYZus{}population}\PY{p}{(}\PY{n}{population}\PY{p}{)}
        
        \PY{c+c1}{\PYZsh{} Sort population and fitness by fitness value}
        \PY{n}{population}\PY{p}{,} \PY{n}{fitness} \PY{o}{=} \PY{n+nb+bp}{self}\PY{o}{.}\PY{n}{\PYZus{}sort\PYZus{}population\PYZus{}fitness}\PY{p}{(}\PY{n}{population}\PY{p}{,} \PY{n}{fitness}\PY{p}{)}
        
        \PY{n}{best\PYZus{}solutions} \PY{o}{=} \PY{p}{[}\PY{n}{fitness}\PY{p}{[}\PY{l+m+mi}{0}\PY{p}{]}\PY{p}{]}
        
        \PY{k}{for} \PY{n}{i} \PY{o+ow}{in} \PY{n+nb}{range}\PY{p}{(}\PY{n+nb+bp}{self}\PY{o}{.}\PY{n}{max\PYZus{}iter}\PY{p}{)}\PY{p}{:}
            \PY{c+c1}{\PYZsh{} 1. Selection}
            \PY{n}{new\PYZus{}population} \PY{o}{=} \PY{n+nb+bp}{self}\PY{o}{.}\PY{n}{\PYZus{}selection\PYZus{}schema}\PY{p}{(}\PY{n}{population}\PY{p}{,} \PY{n}{fitness}\PY{p}{)}
            \PY{n}{new\PYZus{}fitness} \PY{o}{=} \PY{n+nb+bp}{self}\PY{o}{.}\PY{n}{\PYZus{}evaluate\PYZus{}population}\PY{p}{(}\PY{n}{new\PYZus{}population}\PY{p}{)}
            \PY{n}{new\PYZus{}population}\PY{p}{,} \PY{n}{new\PYZus{}fitness} \PY{o}{=} \PY{n+nb+bp}{self}\PY{o}{.}\PY{n}{\PYZus{}sort\PYZus{}population\PYZus{}fitness}\PY{p}{(}\PY{n}{new\PYZus{}population}\PY{p}{,} \PY{n}{new\PYZus{}fitness}\PY{p}{)}
            
            \PY{c+c1}{\PYZsh{} 2. Cross}
            \PY{n}{new\PYZus{}population} \PY{o}{=} \PY{n+nb+bp}{self}\PY{o}{.}\PY{n}{\PYZus{}cross\PYZus{}schema}\PY{p}{(}\PY{n}{new\PYZus{}population}\PY{p}{)}
            
            \PY{c+c1}{\PYZsh{} 3. Mutate}
            \PY{n}{new\PYZus{}population} \PY{o}{=} \PY{n+nb+bp}{self}\PY{o}{.}\PY{n}{\PYZus{}mutation}\PY{p}{(}\PY{n}{new\PYZus{}population}\PY{p}{)}
            
            \PY{c+c1}{\PYZsh{} 4. Evaluation and sorting}
            \PY{n}{new\PYZus{}fitness} \PY{o}{=} \PY{n+nb+bp}{self}\PY{o}{.}\PY{n}{\PYZus{}evaluate\PYZus{}population}\PY{p}{(}\PY{n}{new\PYZus{}population}\PY{p}{)}
            \PY{n}{new\PYZus{}population}\PY{p}{,} \PY{n}{new\PYZus{}fitness} \PY{o}{=} \PY{n+nb+bp}{self}\PY{o}{.}\PY{n}{\PYZus{}sort\PYZus{}population\PYZus{}fitness}\PY{p}{(}\PY{n}{new\PYZus{}population}\PY{p}{,} \PY{n}{new\PYZus{}fitness}\PY{p}{)}
            \PY{n}{population}\PY{p}{,} \PY{n}{fitness} \PY{o}{=} \PY{n+nb+bp}{self}\PY{o}{.}\PY{n}{\PYZus{}elitism}\PY{p}{(}\PY{n}{population}\PY{p}{,} \PY{n}{fitness}\PY{p}{,} \PY{n}{new\PYZus{}population}\PY{p}{,} \PY{n}{new\PYZus{}fitness}\PY{p}{)}
            
            \PY{n}{best\PYZus{}solutions}\PY{o}{.}\PY{n}{append}\PY{p}{(}\PY{n}{fitness}\PY{p}{[}\PY{l+m+mi}{0}\PY{p}{]}\PY{p}{)}
            
            \PY{k}{if} \PY{n}{verbose} \PY{o+ow}{and} \PY{n}{i} \PY{o}{\PYZpc{}} \PY{l+m+mi}{1000} \PY{o}{==} \PY{l+m+mi}{0}\PY{p}{:}
                \PY{n+nb}{print}\PY{p}{(}\PY{l+s+sa}{f}\PY{l+s+s1}{\PYZsq{}}\PY{l+s+s1}{Iteration }\PY{l+s+si}{\PYZob{}}\PY{n}{i}\PY{l+s+si}{\PYZcb{}}\PY{l+s+s1}{/}\PY{l+s+si}{\PYZob{}}\PY{n+nb+bp}{self}\PY{o}{.}\PY{n}{max\PYZus{}iter}\PY{l+s+si}{\PYZcb{}}\PY{l+s+se}{\PYZbs{}t}\PY{l+s+s1}{Best solution: }\PY{l+s+si}{\PYZob{}}\PY{n}{population}\PY{p}{[}\PY{l+m+mi}{0}\PY{p}{]}\PY{l+s+si}{\PYZcb{}}\PY{l+s+se}{\PYZbs{}t}\PY{l+s+s1}{Fitness: }\PY{l+s+si}{\PYZob{}}\PY{n}{fitness}\PY{p}{[}\PY{l+m+mi}{0}\PY{p}{]}\PY{l+s+si}{\PYZcb{}}\PY{l+s+s1}{\PYZsq{}}\PY{p}{)}
        
        \PY{k}{if} \PY{n}{verbose}\PY{p}{:}
            \PY{n+nb}{print}\PY{p}{(}\PY{l+s+sa}{f}\PY{l+s+s1}{\PYZsq{}}\PY{l+s+s1}{Iteration }\PY{l+s+si}{\PYZob{}}\PY{n}{i}\PY{o}{+}\PY{l+m+mi}{1}\PY{l+s+si}{\PYZcb{}}\PY{l+s+s1}{/}\PY{l+s+si}{\PYZob{}}\PY{n+nb+bp}{self}\PY{o}{.}\PY{n}{max\PYZus{}iter}\PY{l+s+si}{\PYZcb{}}\PY{l+s+se}{\PYZbs{}t}\PY{l+s+s1}{Best solution: }\PY{l+s+si}{\PYZob{}}\PY{n}{population}\PY{p}{[}\PY{l+m+mi}{0}\PY{p}{]}\PY{l+s+si}{\PYZcb{}}\PY{l+s+se}{\PYZbs{}t}\PY{l+s+s1}{Fitness: }\PY{l+s+si}{\PYZob{}}\PY{n}{fitness}\PY{p}{[}\PY{l+m+mi}{0}\PY{p}{]}\PY{l+s+si}{\PYZcb{}}\PY{l+s+s1}{\PYZsq{}}\PY{p}{)}
            
            \PY{n}{plt}\PY{o}{.}\PY{n}{plot}\PY{p}{(}\PY{n}{np}\PY{o}{.}\PY{n}{arange}\PY{p}{(}\PY{n+nb+bp}{self}\PY{o}{.}\PY{n}{max\PYZus{}iter} \PY{o}{+} \PY{l+m+mi}{1}\PY{p}{)}\PY{p}{,} \PY{n}{best\PYZus{}solutions}\PY{p}{)}
            \PY{n}{plt}\PY{o}{.}\PY{n}{xlabel}\PY{p}{(}\PY{l+s+s1}{\PYZsq{}}\PY{l+s+s1}{Number of iterations}\PY{l+s+s1}{\PYZsq{}}\PY{p}{)}
            \PY{n}{plt}\PY{o}{.}\PY{n}{ylabel}\PY{p}{(}\PY{l+s+s1}{\PYZsq{}}\PY{l+s+s1}{Fitness value of the best value}\PY{l+s+s1}{\PYZsq{}}\PY{p}{)}
            
            \PY{n}{plt}\PY{o}{.}\PY{n}{show}\PY{p}{(}\PY{p}{)}
        
        \PY{k}{return} \PY{n}{population}\PY{p}{[}\PY{l+m+mi}{0}\PY{p}{]}\PY{p}{,} \PY{n}{fitness}\PY{p}{[}\PY{l+m+mi}{0}\PY{p}{]}
\end{Verbatim}
\end{tcolorbox}

    For this case, we have set cross\_rate \$ = 0.7\$, mutation\_rate
\(= 0.001\) and the stopping criterion as 10000 iterations. For the
\textbf{BLX-\(\alpha\)} crossover we have used a value of
\(\alpha = 0.2\). For the mutation operator, we have set that
\(\mu = 0\) and \(\sigma = 0.7\). With this in mind, let us now declare
an instance of the class and run the algorithm.

    \begin{tcolorbox}[breakable, size=fbox, boxrule=1pt, pad at break*=1mm,colback=cellbackground, colframe=cellborder]
\prompt{In}{incolor}{3}{\boxspacing}
\begin{Verbatim}[commandchars=\\\{\}]
\PY{n}{pop\PYZus{}size} \PY{o}{=} \PY{l+m+mi}{50}
\PY{n}{x\PYZus{}range} \PY{o}{=} \PY{p}{(}\PY{o}{\PYZhy{}}\PY{l+m+mf}{3.}\PY{p}{,} \PY{l+m+mf}{12.11}\PY{p}{)}
\PY{n}{y\PYZus{}range} \PY{o}{=} \PY{p}{(}\PY{l+m+mf}{4.5}\PY{p}{,} \PY{l+m+mf}{5.8}\PY{p}{)}

\PY{n}{ga} \PY{o}{=} \PY{n}{GeneticAlgorithm}\PY{p}{(}\PY{n}{pop\PYZus{}size}\PY{p}{,} \PY{n}{x\PYZus{}range}\PY{p}{,} \PY{n}{y\PYZus{}range}\PY{p}{)}
\end{Verbatim}
\end{tcolorbox}

    \begin{tcolorbox}[breakable, size=fbox, boxrule=1pt, pad at break*=1mm,colback=cellbackground, colframe=cellborder]
\prompt{In}{incolor}{4}{\boxspacing}
\begin{Verbatim}[commandchars=\\\{\}]
\PY{n}{xy}\PY{p}{,} \PY{n}{fitness} \PY{o}{=} \PY{n}{ga}\PY{o}{.}\PY{n}{train\PYZus{}predict}\PY{p}{(}\PY{n}{verbose}\PY{o}{=}\PY{k+kc}{True}\PY{p}{)}

\PY{n+nb}{print}\PY{p}{(}\PY{l+s+sa}{f}\PY{l+s+s1}{\PYZsq{}}\PY{l+s+s1}{Found solution: }\PY{l+s+si}{\PYZob{}}\PY{n}{xy}\PY{l+s+si}{\PYZcb{}}\PY{l+s+se}{\PYZbs{}t}\PY{l+s+s1}{Fitness value: }\PY{l+s+si}{\PYZob{}}\PY{n}{fitness}\PY{l+s+si}{\PYZcb{}}\PY{l+s+s1}{\PYZsq{}}\PY{p}{)}
\end{Verbatim}
\end{tcolorbox}

    \begin{Verbatim}[commandchars=\\\{\}]
Iteration 0/10000       Best solution: [10.1663998   5.32876163]        Fitness:
35.5019752690798
Iteration 1000/10000    Best solution: [12.11        5.52504595]        Fitness:
38.920521529367356
Iteration 2000/10000    Best solution: [12.11        5.72504424]        Fitness:
39.12052072869683
Iteration 3000/10000    Best solution: [12.11        5.72504424]        Fitness:
39.12052072869683
Iteration 4000/10000    Best solution: [12.11        5.72504424]        Fitness:
39.12052072869683
Iteration 5000/10000    Best solution: [12.11        5.72504424]        Fitness:
39.12052072869683
Iteration 6000/10000    Best solution: [12.11        5.72504424]        Fitness:
39.12052072869683
Iteration 7000/10000    Best solution: [12.11        5.72504424]        Fitness:
39.12052072869683
Iteration 8000/10000    Best solution: [12.11        5.72504424]        Fitness:
39.12052072869683
Iteration 9000/10000    Best solution: [12.11        5.72504424]        Fitness:
39.12052072869683
Iteration 10000/10000   Best solution: [12.11        5.72504424]        Fitness:
39.12052072869683
    \end{Verbatim}

    \begin{center}
    \adjustimage{max size={0.9\linewidth}{0.9\paperheight}}{img/genetic_algorithm_5_1.png}
    \end{center}
    { \hspace*{\fill} \\}
    
    \begin{Verbatim}[commandchars=\\\{\}]
Found solution: [12.11        5.72504424]       Fitness value: 39.12052072869683
    \end{Verbatim}

    As we can observe, we have reached a maximum at the ponts
\(x^* = 12.11\), \(y^* = 5.72504424\), where the value of the function
is \(f(x^*, y^*) = 39.12052072869683\).

If we observe the graph, we can see how the the fitness value of the
best chromosome of the population changes as the algorithm progresses.
We see that it starts at around \(35.5\) and it reaches the best value
in less than 2000 iterations. After that, it doesn't improve. Thus, we
could have defined some strategy to stop before.

Something that we should keep in mind is that the algorithm is indeed
stochastic, which means that the best fitness values and how they evolve
depend on the initial values (which are randomly generated). However,
this technique is very powerful, because it can be used to optimize any
kind of function without having any information of the function itself
and its shapes. Also, it can be used to optimize non-continuos
functions, which is a hughe advantage if we compare it to other methods.


    % Add a bibliography block to the postdoc
    
    
    
\end{document}
